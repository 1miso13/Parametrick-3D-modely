

\chapter{Zoznam geometrických operácií}
\label{Priloha:zoznamGeometrickychOperacii}
Geometrické operácie z~ktorých sa skladajú parametrické 3D modely. Tieto operácie sa začínajú názvom operácie, nasledované parametrami v~zátvorkách, ktoré sú oddelené čiarkou.

\section*{Bodové operácie}
\begin{itemize}

% 		Create point on position 0,0,0. Parent - If entered, position is relative, else absolute

\item \textsc{\textbf{Point(string name, expression X, expression Y, expression Z, RRGGBBAA color)}} - Bod s absolútnou pozíciou 
\item \textsc{\textbf{Point(string name, expression X, expression Y, expression Z, Point parent, RRGGBBAA color)}} - Bod s relatívnou pozíciou of zadaného bodu
% 		//-Create point on position XYZ
% 		///	Example:
% 		//		Point("Name of point 1")	//Name can by written with "" or '' or without, Point is on absolute position [0,0,0]
% 		//		Point(NameOfPoint2)		//Name can by written with "" or '' or without
% 		//		Point('NameOfPoint3')	//Name can by written with "" or '' or without
% 		//		Point(NameOfPoint4,1.1,2.2,3.3)	//- Create visible Point on position [1,2,3] 
% 		//		Point(NameOfPoint4,1,2,3,true)	//- Create visible Point on position [1,2,3] 
% 		//		Point(NameOfPoint4,1,2,3,false)	//- Create invisible Point on position [1,2,3]
% 		//		Point(NameOfPoint4,1,2,3,NameOfParentPoint,true)	//- Create visible Point on relative position [1,2,3] from ParentPoint

\item \textsc{\textbf{LinearInterpolationDist(string name, Point from, Point to, expression distance, RRGGBBAA color)}} - Bod medzi dvomi bodmi, so vzdialenosťou od prvého bodu
		 
\item \textsc{\textbf{LinearInterpolationPerc(string name, Point from, Point to, expression percentage, RRGGBBAA color)}} - Bod medzi dvomi bodmi, s percentuálnou vzdialenosťou od prvého bodu
% 		//	Example:
% 		//		LinearInterpolation(PointName, FromPoint, ToPoint, 50%)	//- Create Point in middle of entered points
% 		//		LinearInterpolation(PointName, FromPoint, ToPoint, 25%)	//- Create Point on position with distance 20% of first point and 80% of second point
% 		//		LinearInterpolation(PointName, FromPoint, ToPoint, 100)	//- Create Point on position with distance 100 from first point in angle to second point 

\item \textsc{\textbf{Intersection\_Plane\_Line(string name, Line l, Sufrace s, RRGGBBAA color)}} - Bod v pozícii, kde úsečka pretína plochu %Create Point on position where line l intersect surface s
% 		//	Alternative:
% 		//	Intersection(string name, Line lineName, Sufrace surfaceName)
% 		//	Example:
% 		//		Intersection\_Plane\_Line(PointName,  lineName, surfaceName) //- Create Point on position where Line with name "lineName" intersecting Surface with name "surfaceName"


\item \textsc{\textbf{SurfaceCenterBoundingSquare(string name, Surface s, RRGGBBAA color)}} - Bod v centre plochy %Create point on position in middle of bounding box of entered surface
% 		//Create point on position of middle of entered surface
% 		//	Example:
% 		//		SurfaceCenterBoundingSquare(PointName, Circle)	//- Create Point on center of Circle
% 		//		SurfaceCenterBoundingSquare(PointName, Rectangle)	//- Create Point on middle of Rectangle
% 		//		SurfaceCenter(PointName, Polygon)		//- Create Point on middle of Polygon 
% 		//		SurfaceCenterBoundingSquare(PointName, Polygon)		//- Create Point on middle of Polygon - centroid (sum of points / count of points)

\item \textsc{\textbf{SurfaceCenterAverage(string name, Surface s, RRGGBBAA color) }} - Bod roviny vytvorený spriemerovaním všetkých bodov plochy%Create point on avarage position of all points of entered surface

\item \textsc{\textbf{Centroid(string name, Triangle t, RRGGBBAA color)}} - Bod v ťažisku trojuholníka % Create point on centroid of triangle

\item \textsc{\textbf{Incenter(string name, Triangle t, RRGGBBAA color)}} - Bod v centre vpísaného kruhu trojuholníka%Create point on center of inscribed circle in triangle

\item \textsc{\textbf{Circumcenter(string name, Triangle t, RRGGBBAA color) }} - Bod v centre opísaného kruhu %Create point on center of circumscribed circle over triangle

\item \textsc{\textbf{Orthocenter(string name, Triangle t, RRGGBBAA color)}} - Ortocentrum trojuholníka %Create orthocentrum of triangle

\item \textsc{\textbf{NinePointCenter(string name, Triangle t, RRGGBBAA color)}} - Stred kruhu deviatich bodov trojuholníka %Create point at center of nine points circle of triangle

\item \textsc{\textbf{ObjectCenterBoundingBox(string name, Object3D o, RRGGBBAA color)}} - Stred ohraničujúceho kvádra objektu%Create point on position in middle of bounding box of entered object

\item \textsc{\textbf{ObjectCenterAverage(string name, Object3D o, RRGGBBAA color)}} - Vážený stred objektu, tvorený spriemerovaním všetkých bodov objektu%Create point on avarage position of all points of entered object

\item \textsc{\textbf{LineFirstPoint(string name, Line l, RRGGBBAA color)}} - Počiatočný bod úsečky%Begin point of line

\item \textsc{\textbf{LineSecondPoint(string name, Line l, RRGGBBAA color)}} - Koncový bod úsečky%End point of line

\end{itemize}

\section*{Úsečkové operácie}
		
\begin{itemize}
\item \textsc{\textbf{Line(string name, Point p1, Point p2, RRGGBBAA color)}}  - Vytvorenie úsečky, kde p1 je počiatočný bod a p2 je koncový bod %Create line, where p1 is start point and p2 is end point

\item \textsc{\textbf{LineNormalize(string name, Line l, RRGGBBAA color)}} - Normalizujte čiaru na vzdialenosť 1,0. Počiatočný bod je rovnaký ako počiatočný bod zadanej čiary, konečný bod je vo vzdialenosti 1,0 v smere koncového bodu čiary. %Normalize line to distance 1.0. Beginning point is same as beginning point of entered line, end point is in distance 1.0 in direction to end point of line

\item \textsc{\textbf{LineChangeLengthDist(string name, Line l, expression distance, RRGGBBAA color)}} - Počiatočný bod je rovnaký ako počiatočný bod zadaného riadku, koncový bod je v zadanej vzdialenosti v smere k koncovému bodu riadku%Beginning point is same as beginning point of entered line, end point is in entered distance in direction to end point of line
		
\item \textsc{\textbf{LineChangeLengthPerc(string name, Line l, expression percent, RRGGBBAA color)}} - Počiatočný bod je rovnaký ako počiatočný bod zadaného riadku, koncový bod je v zadanej percentuálnej vzdialenosti v smere k cieľovému bodu riadku %Beginning point is same as beginning point of entered line, end point is in entered percentual distance in direction to end point of line



		
\item \textsc{\textbf{MinLineBetweenLineAndLine(string name, Line l1, Line l2, RRGGBBAA color)}} - Najkratšia úsečka medzi dvomi priamkami%Minimal line between two lines
		
\item \textsc{\textbf{MinLineBetweenPointAndLine(string name, Point p, Line l, RRGGBBAA color)}} - Najkratšia úsečka medzi bodom a priamkou
		
\item \textsc{\textbf{MinLineBetweenPointAndSurface(string name, Point p, Surface s, RRGGBBAA color)}} - Najkratšia úsečka medzi bodom a plochou%Minimal line between point and surface


\item \textsc{\textbf{SurfaceNormal(string name, Surface s, RRGGBBAA color)}}  - Normála plochy%Normal vector of surface



\item \textsc{\textbf{LineRelocationByPoint(string name, Line l, Point p, RRGGBBAA color)}} - Presun úsečky na novú pozíciu. Začiatočný bod je zadaný bod a koncový bod je vo vzdialenosti zadanej čiary rovnakým smerom.%Line is moved to new location. Beginning point is entered point and end point is in distance of entered line with same direction.
\item \textsc{\textbf{CrossProduct(string name, Line l1, Line l2, RRGGBBAA color)}} - Úsečka v smere vektorového súčinu dvoch priamok%Create cross-product of entered lines

\end{itemize}




\section*{Plošné operácie}

\begin{itemize}

\item \textsc{\textbf{RectangleFromLine(string name, Line l, expression width, Point pointOnSurface, type, RRGGBBAA color)}} - Vytvorenie obdĺžniku podľa úsečky, šírky obdĺžniku a bodu na ploche  %Create rectangle surface from line
	
\item \textsc{\textbf{RectangleFromLine(string name, Line l, expression width, Line normal, type, RRGGBBAA color)}} - Vytvorenie obdĺžniku podľa úsečky, šírky obdĺžniku a normály od plochy %Create rectangle surface from line by normal
% //create Rectangle from Line l
% /*type:
% 	0 - width/2 to left, width/2 to right
% 	1 - width to left
% 	2 - width to right
% 	*/
% 	//if normal vector is not perpendicular to line, as normal is used normalized dot product between line and normal vector
% 	//if normal vector is same direction as line normal, exception occure
% 	//If surface point is not on line l, exception occure

\item \textsc{\textbf{Circle(string name, Point center, expression radius, Line lineNormal, RRGGBBAA color)}} - Vytvorenie kruhu podľa zadaného stredu, priemeru a normály%Create circle by center, radius and normal vector
	
\item \textsc{\textbf{Circle(string name, Point center, Point outlinePoint, Point planePoint, RRGGBBAA color)}} - Vytvorenie kruhu od stredu, bodu na okraji kruhu a bodu na ploche %Create circle by 3 points

\item \textsc{\textbf{Triangle(string name, Line l, Point p, RRGGBBAA color)}} - Vytvorenie trojuholníku z úsečky a bodu%Create triangle by line and point
	
\item \textsc{\textbf{Triangle(string name, Point p1, Point p2, Point p3, RRGGBBAA color)}} -  Vytvorenie trojuholníku z troch bodov%Create triangle by 3 points


\item \textsc{\textbf{Rectangle(string name, Point center, expression X, expression Y, expression Roll
% 	/*[0,360]*/
    , Line normal, RRGGBBAA color)}} - Vytvorenie obdĺžniku%Create recangle

\item \textsc{\textbf{Polygon(string name, Point p1, Point p2, Point p3, ..., RRGGBBAA color)}} - Vytvorenie polygónu z ľubovoľného počtu bodov%Create Polygon by connecting points, minimum 3 points.  Multiple points are divided by ';'


\item \textsc{\textbf{Circumscribed(string name, Triangle t, RRGGBBAA color)}}  - Vytvorenie opísanej kružnice trojuholníka %Create circumscribed circle around triangle
	
\item \textsc{\textbf{Inscribed(string name, Triangle t, RRGGBBAA color)}} - Vytvorenie vpísanej kružnice trojuholníka %Create inscribed circle in triangle		
	
\end{itemize}
\section*{Objemové operácie}

\begin{itemize}
\item \textsc{\textbf{Cone(string name, Surface s, expression distance, RRGGBBAA color)}} - Vytvorenie kužeľu s podstavou z ľubovoľného plošného útvaru so zadanou výškou%Create pyramid from surface and distance from center

\item \textsc{\textbf{Cone(string name, Surface s, Point p, RRGGBBAA color)}} - Vytvorenie kužeľu s podstavou z ľubovoľného plošného útvaru a zadaným vrcholom%Create pyramid from surface and specific point

\item \textsc{\textbf{Extrude(string name, Surface s, expression distance, RRGGBBAA color)}}  - Vysunutie plochy do priestoru%Enlarge surface to 3D with distance

%\item SpericalCurvedSurface(string objectName, Surface s, expression distance, RRGGBBAA color)

\item \textsc{\textbf{Sphere(string name, Point center, expression radius, RRGGBBAA color)}} - Guľa zo zadaného bodu a priemeru

\item \textsc{\textbf{BooleanUnion(string name, Object o1, Object o2, RRGGBBAA color)}} - Zjednotenie objektov
\item \textsc{\textbf{BooleanIntersection(string name, Object o1, Object o2, RRGGBBAA color)}} - Prienik objektov%Intersection between objects
\item \textsc{\textbf{BooleanMinus(string name, Object o1, Object o2, RRGGBBAA color)}} - Rozdiel objektov%First object minus second object
\item \textsc{\textbf{BooleanXOR(string name, Object o1, Object o2, RRGGBBAA color)}} - Symetrická diferencia objektov%Symmetric difference between objects
\end{itemize}


\chapter{Množina operácii, tvoriace model hlavy.}
\label{mnozinaoperacii}
\begin{lstlisting}[language=C,frame=tb,numbers=left]
Point(center,0,0,0,00000000);
Sphere(head,center,2:headSize,00FF00FF);
Point(param1,0.5:earSize,3.14/4:earDistance,0.7:earPopOut,00000000);
Point(centerOfLEar,
	0-(sin(earDistance)*(head.radius+earSize*earPopOut)),
	cos(earDistance)*(head.radius+earSize*earPopOut),
	0,center,0000FFFF);
Point(centerOfREar,
	sin(earDistance)*(head.radius+earSize*earPopOut),
	cos(earDistance)*(head.radius+earSize*earPopOut),
	0,center,0000FFFF);
Sphere(leftEar,centerOfLEar,earSize,FF0000FF);
Sphere(rightEar,centerOfREar,earSize,FF0000FF);
Point(paramEyes,3.14/8:eyesWidth,3.14/8:eyesHeight,0.7:eyePopOut,00000000);
Point(paramEyes2,0.2:eyeSize,0,0,00000000);
Point(lEyeCenter,
	sin(eyesWidth)*(head.radius+eyeSize*(0-eyePopOut)),
	cos(eyesWidth)*(head.radius+eyeSize*(0-eyePopOut))*sin(eyesHeight),
	cos(eyesHeight)*(head.radius+eyeSize*(0-eyePopOut))*cos(eyesWidth),
	FF00FFFF);
Point(rEyeCenter,
	0-sin(eyesWidth)*(head.radius+eyeSize*(0-eyePopOut)),
	cos(eyesWidth)*(head.radius+eyeSize*(0-eyePopOut))*sin(eyesHeight),
	cos(eyesHeight)*(head.radius+eyeSize*(0-eyePopOut))*cos(eyesWidth),
	0000FFFF);
Sphere(leftEye,lEyeCenter,eyeSize,FFFF00FF);
Sphere(rightEye,rEyeCenter,eyeSize,FF0FF0FF);

Point(param2,0.7:eyeDotPopOut,eyeSize/2:eyeDotSize,0,00000000);
Point(LookAtPoint,0:lookAtX,0:lookAtY,10:lookAtDistance,00000000);
LinearInterpolationDist(eyeDotLCenter,lEyeCenter,LookAtPoint,eyeSize-eyeDotSize*eyeDotPopOut,00000000);
Sphere(eyeDotL,eyeDotLCenter,eyeDotSize,FF0000FF);
LinearInterpolationDist(eyeDotRCenter,rEyeCenter,LookAtPoint,eyeSize-eyeDotSize*eyeDotPopOut,00000000);
Sphere(eyeDotR,eyeDotRCenter,eyeDotSize,FF0000FF);
\end{lstlisting}{}