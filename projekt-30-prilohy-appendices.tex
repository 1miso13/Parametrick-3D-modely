 \renewcommand\labelitemii{$\square$}
 \renewcommand\labelitemiii{$\circ$}
\chapter{Obsah priloženého média}
\begin{itemize}
\item \textbf{Models/} - Vytvorené modely a obrázky
\item \textbf{Release/} - Spustiteľné súbory
\begin{itemize}
    \item \textbf{ThesisLibrary/}
    \begin{itemize}
        \item \textbf{Thesis.lib} - Knižnica na tvorbu parametrických modelov
    \end{itemize}
    \item \textbf{ParametricModel\_Editor/...} - Aplikácia navrhnutá na tvorbu parametrických modelov
    \item \textbf{GLFW\_Application/...} - Aplikácia s použitím glfw s modelom z príkladu 6.2
    \item \textbf{QtProgram/...} - Aplikácia s použitím Qt z príkladu 6.3
\end{itemize}  
\item \textbf{Source/} - Zdrojové kódy
\begin{itemize}
    \item \textbf{ExternalLibraries/..} - Použité externé knižnice
    \item \textbf{Thesis/..} - Zdrojový kód navrhnutého riešenia
\end{itemize}
\item \textbf{latex/...} Projekt textovej časti
\item Parametricke\_3D\_modely.pdf
\item README.txt
\end{itemize}

\chapter{Rozhranie navrhnutej knižnice}
\begin{lstlisting}[numbers=left,language=c++]
std::vector <Operation*> OperationsVec; //Operations stack 
std::map<std::string, Operation*> OperationMap; //Map for easier acces to operations

std::vector <Object::GeometricObject*> Objects; //Generated objects of parametric model
std::map<std::string, Object::GeometricObject*> ObjectMap; // Map for objects of parametric model

ParamRef paramRef //Model parameter referencies
ParametricModel() //Constructor
~ParametricModel() //Destructor

Object::GeometricObject* GetObject(std::string objectName)
void BuildModel() //Generate model objects from operations. Call this to apply modifications
void RemoveOperation(size_t index) //Remove operation from specific index
void AddOperation(Operation *c) //Add operation to model
void InsertOperation(size_t index,Operation *c) //Insert operation to specific position
void DeleteModel() //Reset model to default state
bool AddOperations(std::string s) //Add multiple operations to model
bool AddOperation(std::string s) //Add operation to model

void resetTest() //Test initializer. Need to be called before every test 
bool TestOperation(std::string s) //Test if operation is valid

bool AddReferenceParam(std::string ParamRefName, std::string refObjectName, size_t paramIndex) //Add model parameter
bool AddReferenceParamWithoutValidCheck(std::string ParamRefName, std::string refObjectName, size_t paramIndex) 
bool TestRefParam(std::string ParamRefName, std::string refObjectName, size_t paramIndex) //Test if parameter can be added to object
void RemoveReferenceParam(std::string ParamRefName) //Remove model parameter
void RemoveReferenceParam(std::string refObjectName, size_t paramIndex) //Remove model parameter from specific object  
void ReplaceOperation(size_t index, Operation *c) //Replace operation

void SetRefValue(std::string ObjectName, size_t paramindex, std::string value) //Set value of operation parameter 
void SetRefValue(std::string RefName,float value) //Set value of parameter

double GetObjectValue(std::string s,bool* Err = NULL)  //Get value in string form

void InitRenderer() //Initialize renderer
void Draw(int x, int y, float fov, int width, int height) //Draw  with OpenGL

void SetRendererCameraPosition(float X, float Y, float Z)
void SetRendererCameraRotation(float Pitch, float Yaw, float 
void setRendererAmbientStrength(float ambientStrength)

void Save(std::string filePath) //Save parametric model to file
bool Load(std::string filePath) //Load parametric model to file
void SaveOBJ(std::string filePath) //Export model to OBJ file

void setTime(unsigned long miliseconds) //Set timer to selected time
void useManualTimer(bool enable) //Enable or disable timer
void SetBackgroundColor(float R, float G, float B, float A) 
void SetBackgroundColor(signed char R, signed char G, signed char B, signed char A)
\end{lstlisting}

\chapter{Zoznam geometrických operácií}
\label{Priloha:zoznamGeometrickychOperacii}
Geometrické operácie, z~ktorých sa skladajú parametrické 3D modely. Tieto operácie sa začínajú názvom operácie, nasledované parametrami v~zátvorkách, ktoré sú oddelené čiarkou.

\section*{Bodové operácie}
\begin{itemize}

% 		Create point on position 0,0,0. Parent - If entered, position is relative, else absolute

\item \textsc{\textbf{Point(string name, expression X, expression Y, expression Z, RRGGBBAA color)}} - Bod s~absolútnou pozíciou 
\item \textsc{\textbf{Point(string name, expression X, expression Y, expression Z, Point parent, RRGGBBAA color)}} - Bod s~relatívnou pozíciou of zadaného bodu
% 		//-Create point on position XYZ
% 		///	Example:
% 		//		Point("Name of point 1")	//Name can by written with "" or '' or without, Point is on absolute position [0,0,0]
% 		//		Point(NameOfPoint2)		//Name can by written with "" or '' or without
% 		//		Point('NameOfPoint3')	//Name can by written with "" or '' or without
% 		//		Point(NameOfPoint4,1.1,2.2,3.3)	//- Create visible Point on position [1,2,3] 
% 		//		Point(NameOfPoint4,1,2,3,true)	//- Create visible Point on position [1,2,3] 
% 		//		Point(NameOfPoint4,1,2,3,false)	//- Create invisible Point on position [1,2,3]
% 		//		Point(NameOfPoint4,1,2,3,NameOfParentPoint,true)	//- Create visible Point on relative position [1,2,3] from ParentPoint

\item \textsc{\textbf{LinearInterpolationDist(string name, Point from, Point to, expression distance, RRGGBBAA color)}} - Bod medzi dvomi bodmi, so vzdialenosťou od prvého bodu
		 
\item \textsc{\textbf{LinearInterpolationPerc(string name, Point from, Point to, expression percentage, RRGGBBAA color)}} - Bod medzi dvomi bodmi, s~percentuálnou vzdialenosťou od prvého bodu
% 		//	Example:
% 		//		LinearInterpolation(PointName, FromPoint, ToPoint, 50%)	//- Create Point in middle of entered points
% 		//		LinearInterpolation(PointName, FromPoint, ToPoint, 25%)	//- Create Point on position with distance 20% of first point and 80% of second point
% 		//		LinearInterpolation(PointName, FromPoint, ToPoint, 100)	//- Create Point on position with distance 100 from first point in angle to second point 

\item \textsc{\textbf{Intersection\_Plane\_Line(string name, Line l, Sufrace s, RRGGBBAA color)}} - Bod v~pozícii, kde úsečka pretína plochu %Create Point on position where line l intersect surface s
% 		//	Alternative:
% 		//	Intersection(string name, Line lineName, Sufrace surfaceName)
% 		//	Example:
% 		//		Intersection\_Plane\_Line(PointName,  lineName, surfaceName) //- Create Point on position where Line with name "lineName" intersecting Surface with name "surfaceName"


\item \textsc{\textbf{SurfaceCenterBoundingSquare(string name, Surface s, RRGGBBAA color)}} - Bod v~centre plochy %Create point on position in middle of bounding box of entered surface
% 		//Create point on position of middle of entered surface
% 		//	Example:
% 		//		SurfaceCenterBoundingSquare(PointName, Circle)	//- Create Point on center of Circle
% 		//		SurfaceCenterBoundingSquare(PointName, Rectangle)	//- Create Point on middle of Rectangle
% 		//		SurfaceCenter(PointName, Polygon)		//- Create Point on middle of Polygon 
% 		//		SurfaceCenterBoundingSquare(PointName, Polygon)		//- Create Point on middle of Polygon - centroid (sum of points / count of points)

\item \textsc{\textbf{SurfaceCenterAverage(string name, Surface s, RRGGBBAA color) }} - Bod roviny vytvorený spriemerovaním všetkých bodov plochy%Create point on avarage position of all points of entered surface

\item \textsc{\textbf{Centroid(string name, Triangle t, RRGGBBAA color)}} - Bod v~ťažisku trojuholníka % Create point on centroid of triangle

\item \textsc{\textbf{Incenter(string name, Triangle t, RRGGBBAA color)}} - Bod v~centre vpísaného kruhu trojuholníka%Create point on center of inscribed circle in triangle

\item \textsc{\textbf{Circumcenter(string name, Triangle t, RRGGBBAA color) }} - Bod v~centre opísaného kruhu %Create point on center of circumscribed circle over triangle

\item \textsc{\textbf{Orthocenter(string name, Triangle t, RRGGBBAA color)}} - Ortocentrum trojuholníka %Create orthocentrum of triangle

\item \textsc{\textbf{NinePointCenter(string name, Triangle t, RRGGBBAA color)}} - Stred kruhu deviatich bodov trojuholníka %Create point at center of nine points circle of triangle

\item \textsc{\textbf{ObjectCenterBoundingBox(string name, Object3D o, RRGGBBAA color)}} - Stred ohraničujúceho kvádra objektu%Create point on position in middle of bounding box of entered object

\item \textsc{\textbf{ObjectCenterAverage(string name, Object3D o, RRGGBBAA color)}} - Vážený stred objektu, tvorený spriemerovaním všetkých bodov objektu%Create point on avarage position of all points of entered object

\item \textsc{\textbf{LineFirstPoint(string name, Line l, RRGGBBAA color)}} - Počiatočný bod úsečky%Begin point of line

\item \textsc{\textbf{LineSecondPoint(string name, Line l, RRGGBBAA color)}} - Koncový bod úsečky%End point of line

\end{itemize}

\section*{Úsečkové operácie}
		
\begin{itemize}
\item \textsc{\textbf{Line(string name, Point p1, Point p2, RRGGBBAA color)}}  - Vytvorenie úsečky, kde p1 je počiatočný bod a p2 je koncový bod %Create line, where p1 is start point and p2 is end point

\item \textsc{\textbf{LineNormalize(string name, Line l, RRGGBBAA color)}} - Normalizujte čiaru na vzdialenosť 1,0. Počiatočný bod je rovnaký ako počiatočný bod zadanej čiary, konečný bod je vo vzdialenosti 1,0 v~smere koncového bodu čiary. %Normalize line to distance 1.0. Beginning point is same as beginning point of entered line, end point is in distance 1.0 in direction to end point of line

\item \textsc{\textbf{LineChangeLengthDist(string name, Line l, expression distance, RRGGBBAA color)}} - Počiatočný bod je rovnaký ako počiatočný bod zadaného riadku, koncový bod je v~zadanej vzdialenosti v~smere k~koncovému bodu riadku%Beginning point is same as beginning point of entered line, end point is in entered distance in direction to end point of line
		
\item \textsc{\textbf{LineChangeLengthPerc(string name, Line l, expression percent, RRGGBBAA color)}} - Počiatočný bod je rovnaký ako počiatočný bod zadaného riadku, koncový bod je v~zadanej percentuálnej vzdialenosti v~smere k~cieľovému bodu riadku %Beginning point is same as beginning point of entered line, end point is in entered percentual distance in direction to end point of line



		
\item \textsc{\textbf{MinLineBetweenLineAndLine(string name, Line l1, Line l2, RRGGBBAA color)}} - Najkratšia úsečka medzi dvomi priamkami%Minimal line between two lines
		
\item \textsc{\textbf{MinLineBetweenPointAndLine(string name, Point p, Line l, RRGGBBAA color)}} - Najkratšia úsečka medzi bodom a priamkou
		
\item \textsc{\textbf{MinLineBetweenPointAndSurface(string name, Point p, Surface s, RRGGBBAA color)}} - Najkratšia úsečka medzi bodom a plochou%Minimal line between point and surface


\item \textsc{\textbf{SurfaceNormal(string name, Surface s, RRGGBBAA color)}}  - Normála plochy%Normal vector of surface



\item \textsc{\textbf{LineRelocationByPoint(string name, Line l, Point p, RRGGBBAA color)}} - Presun úsečky na novú pozíciu. Začiatočný bod je zadaný bod a koncový bod je vo vzdialenosti zadanej čiary rovnakým smerom.%Line is moved to new location. Beginning point is entered point and end point is in distance of entered line with same direction.
\item \textsc{\textbf{CrossProduct(string name, Line l1, Line l2, RRGGBBAA color)}} - Úsečka v~smere vektorového súčinu dvoch priamok%Create cross-product of entered lines

\end{itemize}




\section*{Plošné operácie}

\begin{itemize}

\item \textsc{\textbf{RectangleFromLine(string name, Line l, expression width, Point pointOnSurface, type, RRGGBBAA color)}} -~Vytvorenie obdĺžniku podľa úsečky, šírky obdĺžniku a bodu na ploche  %Create rectangle surface from line
	
\item \textsc{\textbf{RectangleFromLine(string name, Line l, expression width, Line normal, type, RRGGBBAA color)}} -~Vytvorenie obdĺžniku podľa úsečky, šírky obdĺžniku a normály od plochy %Create rectangle surface from line by normal
% //create Rectangle from Line l
% /*type:
% 	0 - width/2 to left, width/2 to right
% 	1 - width to left
% 	2 - width to right
% 	*/
% 	//if normal vector is not perpendicular to line, as normal is used normalized dot product between line and normal vector
% 	//if normal vector is same direction as line normal, exception occure
% 	//If surface point is not on line l, exception occure

\item \textsc{\textbf{Circle(string name, Point center, expression radius, Line lineNormal, \newline RRGGBBAA color)}} -~Vytvorenie kruhu podľa zadaného stredu, priemeru a normály%Create circle by center, radius and normal vector
	
\item \textsc{\textbf{Circle(string name, Point center, Point outlinePoint, Point planePoint, RRGGBBAA color)}} -~Vytvorenie kruhu od stredu, bodu na okraji kruhu a bodu na ploche %Create circle by 3 points

\item \textsc{\textbf{Triangle(string name, Line l, Point p, RRGGBBAA color)}} -~Vytvorenie trojuholníku z~úsečky a bodu%Create triangle by line and point
	
\item \textsc{\textbf{Triangle(string name, Point p1, Point p2, Point p3, RRGGBBAA color)}} -~Vytvorenie trojuholníku z~troch bodov%Create triangle by 3 points


\item \textsc{\textbf{Rectangle(string name, Point center, expression X, expression Y, expression Roll
% 	/*[0,360]*/
    , Line normal, RRGGBBAA color)}} -~Vytvorenie obdĺžniku%Create recangle

\item \textsc{\textbf{Polygon(string name, Point p1, Point p2, Point p3, ..., RRGGBBAA color)}} -~Vytvorenie polygónu z~ľubovoľného počtu bodov%Create Polygon by connecting points, minimum 3 points.  Multiple points are divided by ';'


\item \textsc{\textbf{Circumscribed(string name, Triangle t, RRGGBBAA color)}}  -~Vytvorenie opísanej kružnice trojuholníka %Create circumscribed circle around triangle
	
\item \textsc{\textbf{Inscribed(string name, Triangle t, RRGGBBAA color)}} -~Vytvorenie vpísanej kružnice trojuholníka %Create inscribed circle in triangle		
	
\end{itemize}
\section*{Objemové operácie}

\begin{itemize}
\item \textsc{\textbf{Cone(string name, Surface s, expression distance, RRGGBBAA color)}} \newline-~Vytvorenie kužeľu s~podstavou z~ľubovoľného plošného útvaru so zadanou výškou%Create pyramid from surface and distance from center

\item \textsc{\textbf{Cone(string name, Surface s, Point p, RRGGBBAA color)}} -~Vytvorenie kužeľu s~podstavou z~ľubovoľného plošného útvaru a zadaným vrcholom%Create pyramid from surface and specific point

\item \textsc{\textbf{Extrude(string name, Surface s, expression distance, RRGGBBAA color)}}  -~Vysunutie plochy do priestoru%Enlarge surface to 3D with distance

%\item SpericalCurvedSurface(string objectName, Surface s, expression distance, RRGGBBAA color)

\item \textsc{\textbf{Sphere(string name, Point center, expression radius, RRGGBBAA color)}} -~Guľa zo zadaného bodu a priemeru

\item \textsc{\textbf{BooleanUnion(string name, Object o1, Object o2, RRGGBBAA color)}} \newline-~Zjednotenie objektov
\item \textsc{\textbf{BooleanIntersection(string name, Object o1, Object o2, RRGGBBAA color)}} -~Prienik objektov%Intersection between objects
\item \textsc{\textbf{BooleanMinus(string name, Object o1, Object o2, RRGGBBAA color)}} \newline-~Rozdiel objektov%First object minus second object
\item \textsc{\textbf{BooleanXOR(string name, Object o1, Object o2, RRGGBBAA color)}} \newline-~Symetrická diferencia objektov%Symmetric difference between objects
\end{itemize}


\chapter{Množina operácii, tvoriace model hlavy.}
\label{mnozinaoperacii}
Z týchto operácii je tvorený model hlavy, ktorý je zobrazený na obrázku \ref{fig:headModel}.

\begin{lstlisting}[language=C,frame=tb,numbers=left]
Point(center,0,0,0,00000000);
Sphere(head,center,2:headSize,00FF00FF);
Point(param1,0.5:earSize,3.14/4:earDistance,0.7:earPopOut,00000000);
Point(centerOfLEar,
	0-(sin(earDistance)*(head.radius+earSize*earPopOut)),
	cos(earDistance)*(head.radius+earSize*earPopOut),
	0,center,0000FFFF);
Point(centerOfREar,
	sin(earDistance)*(head.radius+earSize*earPopOut),
	cos(earDistance)*(head.radius+earSize*earPopOut),
	0,center,0000FFFF);
Sphere(leftEar,centerOfLEar,earSize,FF0000FF);
Sphere(rightEar,centerOfREar,earSize,FF0000FF);
Point(paramEyes,3.14/8:eyesWidth,3.14/8:eyesHeight,0.7:eyePopOut,00000000);
Point(paramEyes2,0.2:eyeSize,0,0,00000000);
Point(lEyeCenter,
	sin(eyesWidth)*(head.radius+eyeSize*(0-eyePopOut)),
	cos(eyesWidth)*(head.radius+eyeSize*(0-eyePopOut))*sin(eyesHeight),
	cos(eyesHeight)*(head.radius+eyeSize*(0-eyePopOut))*cos(eyesWidth),
	FF00FFFF);
Point(rEyeCenter,
	0-sin(eyesWidth)*(head.radius+eyeSize*(0-eyePopOut)),
	cos(eyesWidth)*(head.radius+eyeSize*(0-eyePopOut))*sin(eyesHeight),
	cos(eyesHeight)*(head.radius+eyeSize*(0-eyePopOut))*cos(eyesWidth),
	0000FFFF);
Sphere(leftEye,lEyeCenter,eyeSize,FFFF00FF);
Sphere(rightEye,rEyeCenter,eyeSize,FF0FF0FF);

Point(param2,0.7:eyeDotPopOut,eyeSize/2:eyeDotSize,0,00000000);
Point(LookAtPoint,0:lookAtX,0:lookAtY,10:lookAtDistance,00000000);
LinearInterpolationDist(eyeDotLCenter,lEyeCenter,LookAtPoint,eyeSize-eyeDotSize*eyeDotPopOut,00000000);
Sphere(eyeDotL,eyeDotLCenter,eyeDotSize,FF0000FF);
LinearInterpolationDist(eyeDotRCenter,rEyeCenter,LookAtPoint,eyeSize-eyeDotSize*eyeDotPopOut,00000000);
Sphere(eyeDotR,eyeDotRCenter,eyeDotSize,FF0000FF);
\end{lstlisting}{}




\chapter{Operácie tvoriace model motora}
\label{priloha:motor}

\begin{lstlisting}[numbers=left]
Point(parameters,1.7:offset,0.5:cylinder_height,0.75:cylinder_radius,00000000);
Point(parameters1,2:piston_height,0.2:piston_radius,0,00000000);
Point(parameters2,0.3:Upperdeadcenterpiston_height,2,0.5:A_radius, 00000000);
Point(parameters3,50:speed,2:ASDF,0.15:circle,00000000);
Point(parameters4,speed*(0-time)/10000:rotation,2,0,00000000);
Point(base,0,-1,0,00000000);
Point(pointUP,0,1,0,base,00000000);
Line(lineUP,base,pointUP,00000000);
Point(engine_cylinder0,offset*1.5,0,0,base,00000000);
Point(engine_cylinder1,offset*0.5,0,0,base,00000000);
Point(engine_cylinder2,offset*-0.5,0,0,base,00000000);
Point(engine_cylinder3,offset*-1.5,0,0,base,00000000);
Point(pointLeft0,piston_radius,0,0,engine_cylinder0,00000000);
Point(pointRight0,0-piston_radius,0,0,engine_cylinder0,00000000);
Point(pointLeft1,piston_radius,0,0,engine_cylinder1,00000000);
Point(pointRight1,0-piston_radius,0,0,engine_cylinder1,00000000);
Point(pointLeft2,piston_radius,0,0,engine_cylinder2,00000000);
Point(pointRight2,0-piston_radius,0,0,engine_cylinder2,00000000);
Point(pointLeft3,piston_radius,0,0,engine_cylinder3,00000000);
Point(pointRight3,0-piston_radius,0,0,engine_cylinder3,00000000);
Line(vectorRight,pointLeft0,base,00000000);
Line(vectorLeft,base,pointLeft0,00000000);
Circle(Base_circle0,pointLeft0,piston_height/2,vectorLeft,00000000);
Circle(Base_circle1,pointLeft1,piston_height/2,vectorLeft,00000000);
Circle(Base_circle2,pointLeft2,piston_height/2,vectorLeft,00000000);
Circle(Base_circle3,pointLeft3,piston_height/2,vectorLeft,00000000);
Circle(Base_circleR0,pointRight0,piston_height/2,vectorRight,00000000);
Circle(Base_circleR1,pointRight1,piston_height/2,vectorRight,00000000);
Circle(Base_circleR2,pointRight2,piston_height/2,vectorRight,00000000);
Circle(Base_circleR3,pointRight3,piston_height/2,vectorRight,00000000);
Extrude(BaseR0,Base_circle0, circle,00FFFFFF);
Extrude(BaseR1,Base_circle1, circle,00FFFFFF);
Extrude(BaseR2,Base_circle2, circle,00FFFFFF);
Extrude(BaseR3,Base_circle3, circle,00FFFFFF);
Extrude(BaseL0,Base_circleR0,circle,00FFFFFF);
Extrude(BaseL1,Base_circleR1,circle,00FFFFFF);
Extrude(BaseL2,Base_circleR2,circle,00FFFFFF);
Extrude(BaseL3,Base_circleR3,circle,00FFFFFF);
Point(wing0,0,cos(rotation)*(piston_height/2-piston_radius),sin(rotation)*(piston_height/2-piston_radius),engine_cylinder0,000000FF);
Point(wing1,0,cos(rotation+3.141598)*(piston_height/2-piston_radius),sin(rotation+3.141598)*(piston_height/2-piston_radius),engine_cylinder1,000000FF);
Point(wing2,0,cos(rotation+3.141598)*(piston_height/2-piston_radius),sin(rotation+3.141598)*(piston_height/2-piston_radius),engine_cylinder2,000000FF);
Point(wing3,0,cos(rotation)*(piston_height/2-piston_radius),sin(rotation)*(piston_height/2-piston_radius),engine_cylinder3,000000FF);
Point(CylinderBase0,0,(sqrt(piston_height^2-(wing0.Z-engine_cylinder0.Z)^2))+wing0.Y-engine_cylinder0.Y,0,engine_cylinder0,00FF00FF);
Point(CylinderBase1,0,(sqrt(piston_height^2-(wing1.Z-engine_cylinder1.Z)^2))+wing1.Y-engine_cylinder1.Y,0,engine_cylinder1,00FF00FF);
Point(CylinderBase2,0,(sqrt(piston_height^2-(wing2.Z-engine_cylinder2.Z)^2))+wing2.Y-engine_cylinder2.Y,0,engine_cylinder2,00FF00FF);
Point(CylinderBase3,0,(sqrt(piston_height^2-(wing3.Z-engine_cylinder3.Z)^2))+wing3.Y-engine_cylinder3.Y,0,engine_cylinder3,00FF00FF);
Circle(CylinderHead0_circle,CylinderBase0,cylinder_radius,lineUP,00000000);
Circle(CylinderHead1_circle,CylinderBase1,cylinder_radius,lineUP,00000000);
Circle(CylinderHead2_circle,CylinderBase2,cylinder_radius,lineUP,00000000);
Circle(CylinderHead3_circle,CylinderBase3,cylinder_radius,lineUP,00000000);
Extrude(CylinderHead0,CylinderHead0_circle,cylinder_height,FF00FFFF);
Extrude(CylinderHead1,CylinderHead1_circle,cylinder_height,FF00FFFF);
Extrude(CylinderHead2,CylinderHead2_circle,cylinder_height,FF00FFFF);
Extrude(CylinderHead3,CylinderHead3_circle,cylinder_height,FF00FFFF);
Line(piston_line0,wing0,CylinderBase0,00000000);
Line(piston_line1,wing1,CylinderBase1,00000000);
Line(piston_line2,wing2,CylinderBase2,00000000);
Line(piston_line3,wing3,CylinderBase3,00000000);
Circle(piston_circle0,wing0,piston_radius,piston_line0,00000000);
Circle(piston_circle1,wing1,piston_radius,piston_line1,00000000);
Circle(piston_circle2,wing2,piston_radius,piston_line2,00000000);
Circle(piston_circle3,wing3,piston_radius,piston_line3,00000000);
Extrude(piston0,piston_circle0,piston_height,FFFF00FF);
Extrude(piston1,piston_circle1,piston_height,FFFF00FF);
Extrude(piston2,piston_circle2,piston_height,FFFF00FF);
Extrude(piston3,piston_circle3,piston_height,FFFF00FF);
Point(crankshaft0__point,piston_radius+circle,0,0,wing0,00000000);
Point(crankshaft1__point,piston_radius+circle,0,0,wing1,00000000);
Point(crankshaft2__point,piston_radius+circle,0,0,wing2,00000000);
Point(crankshaft3__point,piston_radius+circle,0,0,wing3,00000000);
Circle(crankshaft0_circle,crankshaft0__point,piston_radius,vectorLeft,0000FF00);
Circle(crankshaft1_circle,crankshaft1__point,piston_radius,vectorLeft,0000FF00);
Circle(crankshaft2_circle,crankshaft2__point,piston_radius,vectorLeft,0000FF00);
Circle(crankshaft3_circle,crankshaft3__point,piston_radius,vectorLeft,0000FF00);
Extrude(crankshaft0_cylinder,crankshaft0_circle, 0-((piston_radius+circle)*2),00FFFFFF);
Extrude(crankshaft1_cylinder,crankshaft1_circle, 0-((piston_radius+circle)*2),00FFFFFF);
Extrude(crankshaft2_cylinder,crankshaft2_circle, 0-((piston_radius+circle)*2),00FFFFFF);
Extrude(crankshaft3_cylinder,crankshaft3_circle, 0-((piston_radius+circle)*2),00FFFFFF);
Point(blockCenter,0,piston_height/2,cylinder_radius,base,00000000);
Rectangle(blockBase,blockCenter,offset*2.5,cylinder_radius,0,lineUP,00000000);
Extrude(block,blockBase, cylinder_height+piston_height+Upperdeadcenterpiston_height*2,FFFFFF00);
Circle(CutOff0_circle,engine_cylinder0,cylinder_radius,lineUP,00000000);
Circle(CutOff1_circle,engine_cylinder1,cylinder_radius,lineUP,00000000);
Circle(CutOff2_circle,engine_cylinder2,cylinder_radius,lineUP,00000000);
Circle(CutOff3_circle,engine_cylinder3,cylinder_radius,lineUP,00000000);
Extrude(CutOff0,CutOff0_circle,cylinder_height+piston_height*1.5+Upperdeadcenterpiston_height,00000000);
Extrude(CutOff1,CutOff1_circle,cylinder_height+piston_height*1.5+Upperdeadcenterpiston_height,00000000);
Extrude(CutOff2,CutOff2_circle,cylinder_height+piston_height*1.5+Upperdeadcenterpiston_height,00000000);
Extrude(CutOff3,CutOff3_circle,cylinder_height+piston_height*1.5+Upperdeadcenterpiston_height,00000000);
BooleanUnion(union01,CutOff0,CutOff1,00000000);
BooleanUnion(union23,CutOff2,CutOff3,00000000);
BooleanUnion(union0123,union01,union23,0000FF00);
BooleanMinus(blockMinus,block,union0123,888888FF);
\end{lstlisting}